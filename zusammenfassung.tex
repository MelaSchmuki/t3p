% Formelsammlung Numerische Methoden der Elektrotechnik
%
% Geschrieben im SS 2014 an der TU München
% von Markus Hofbauer, Kevin Meyer und Benedikt Schmidt für LaTeX4EI
% based on template from www.latex4ei.de
% Kontakt: latex@kevin-meyer.de oder via Kontaktformular auf http://latex4ei.de
% Aktuelle Versionen auf https://makeappdev.github.io/TUM-Projekte/

% Dokumenteinstellungen
% ======================================================================
\documentclass[german]{latex4ei/latex4ei_sheet}
\usepackage[european]{circuitikz}
\usepackage{tabularx}
\usepackage{amsmath}
\usepackage{xcolor}
\usepackage{multirow}
\usetikzlibrary{arrows, calc, intersections}
\usepackage{calc}

% Für code
\definecolor{COMMENTGREEN}{HTML}{228B22}
\definecolor{MATLABBACKGROUND}{HTML}{FCFCDC}
\lstset{ %
language=Matlab,						% choose the language of the code
%basicstyle=\ttfamily,					% the size of the fonts that are used for the code
%emphstyle=\color{yellow}\ttfalily,
%keywordstyle=\color{blue}\ttfamily,
%stringstyle=\color{magenta}\ttfamily,
%commentstyle=\color{COMMENTGREEN}\ttfamily,
xleftmargin=10.3pt,						% distance to margin left
xrightmargin=-3pt,						% distance to margin right
%linewidth=\widthof{\begin{sectionbox}}}		% this would be better instead of left and right margin
%aboveskip=0.6\baselineskip,
%belowskip=0\baselineskip,
numbers=left,                   		% where to put the line-numbers
%framexleftmargin=1.5em,					% distance to xleftmargin
%numberstyle=\ttfamily\footnotesize,		% the size of the fonts that are used for the line-numbers
%stepnumber=1,							% the step between two line-numbers. If it is 1 each line will be numbered
%numbersep=5pt,							% how far the line-numbers are from the code
%backgroundcolor=\color{MATLABBACKGROUND},	% choose the background color. You must add \usepackage{color}
%showspaces=false,						% show spaces adding particular underscores
%showstringspaces=false,					% underline spaces within strings
%showtabs=false,							% show tabs within strings adding particular underscores
%frame=single,							% adds a frame around the code (Box) (Für top und bottom rule set option to "lines")
%rulecolor=\color{gray},					% color of framebox rule
%tabsize=4,								% sets default tabsize to 2 spaces
%captionpos=b,							% sets the caption-position to bottom
breaklines=true,						% sets automatic line breaking
breakatwhitespace=false,				% sets if automatic breaks should only happen at whitespace
escapeinside={\%*}{*)}					% if you want to add a comment within your code
}

% tabularx definition
\newcolumntype{C}{>{\centering\arraybackslash}X}
\newcolumntype{L}{@{\extracolsep\fill}X}

% SI-Zahlen mit Komma als Dezimaltrenner
\sisetup{locale=DE}

% SI-Einheiten
\DeclareSIUnit\voltampere{VA}
\DeclareSIUnit\var{Var}
\DeclareSIUnit\newtonmeter{Nm}
\DeclareSIUnit\voltsecond{Vs}
\DeclareSIUnit\amperesecond{As}

\DeclareMathOperator*{\argmin}{arg\,min}
\DeclareMathOperator{\cond}{cond}
\DeclareMathOperator{\rang}{rang}

\DeclareMathOperator{\Bild}{Bild}
\DeclareMathOperator{\defect}{def}

% Title
\title{T3p: Elektrodynamik}
\author{Marcus Müller}
\myemail{marcus.mueller@physik.uni-muenchen.de}
\mywebsite{}

% Dokumentbeginn
% ======================================================================
\begin{document}

\IfFileExists{git.id}{\input{git.id}}{}
\ifdefined\GitRevision\mydate{\GitNiceDate\ (git \GitRevision)}\fi

\maketitle
\section{Mathematische Grundlagen}

\begin{sectionbox}
	\begin{itemize}
		\item \textbf{Wie lauten Skalar- und Vektorprodukt in Indexnotation?} 
		\begin{itemize}
			\item Skalarprodukt: $\vec{x}\cdot\vec{y}=x^iy_i$ 
			\item Vektorprodukt/Kreuzprodukt: $\vec{x}\times\vec{y}=\varepsilon_{ijk}a_ib_j\vec{e}_k$
		\end{itemize}
		\item \textbf{Was ist das Ergebnis von $\partial_ir_j$?}
		\begin{equation}
			\partial_ir_j=\left(
			\begin{matrix}
  				\partial_1r_1 & ... &\partial_1r_m \\
 				... &... & ...\\
 				\partial_nr_1 &... & \partial_nr_m
			\end{matrix}\right)
		\end{equation}
		\item \textbf{Was ergibt die Kontraktion $\varepsilon_{ijk}\varepsilon_{klm}$ und wie können Sie sich diese merken?}
		\begin{equation}
			\varepsilon_{ijk}\varepsilon_{klm}=\delta_{jl}\delta_{km}-\delta_{km}\delta_{kl}
		\end{equation}
		
		\item \textbf{Wie definieren Sie die Menge, die eine Kugel K um einen\\
		 beliebigen Punkt beschreibt? Wie definieren sie deren Rand $\partial K$? Wie lauten die zugehörigen Parametrisierungen}\\
		$K: (x−x_M)^2+(y−y_M)^2+(z−z_M)^2=r^2$
		\colorbox{red}{noch bearbeiten}
		\item \textbf{Wie lauten die Mengen und die Parametrisierungen der letzten Frage für einen Zylinder Z um eine beliebige zu $e_z$ parallele Achse? Wie lauten sie für eine zur xy-Ebene parallele Kreisscheibe S mit beliebigem Zentrum?}\\
		\colorbox{red}{noch bearbeiten}
		\item \textbf{Was ist die Bedeutung der Divergenz und der Rotation eines Vektorfeldes?}
			\begin{itemize}
				\item Divergenz: $\nabla$
			\end{itemize}
		\item \textbf{Wie konstruiert man Gradient, Divergenz, Rotation und Laplace-Operator in
krummlinigen Koordinaten?}
		\item \textbf{Wie berechnet man ein skalares bzw. vektorielles Kurvenintegral?}
		\item \textbf{Wie berechnet man ein skalares bzw. vektorielles Flächenintegral in kartesischen, Zylinder- oder Kugelkoordinaten?}
		\item \textbf{Wie lauten der Satz von Gauß und der Satz von Stokes?}
		\item \textbf{Was sind die Eigenschaften der Delta-Distribution?}
		\item \textbf{Wie steht der Laplace-Operator mit der Delta-Distribution in Verbindung?}
		\item \textbf{Was besagt das Helmholtz Theorem?}
	\end{itemize}	
\end{sectionbox}
\section{Maxwell-Gleichungen}
\tiny{based on template from www.latex4ei.de}

\end{document}

% Formelsammlung Numerische Methoden der Elektrotechnik
%
% Geschrieben im SS 2014 an der TU München
% von Markus Hofbauer, Kevin Meyer und Benedikt Schmidt für LaTeX4EI
% based on template from www.latex4ei.de
% Kontakt: latex@kevin-meyer.de oder via Kontaktformular auf http://latex4ei.de
% Aktuelle Versionen auf https://makeappdev.github.io/TUM-Projekte/

% Dokumenteinstellungen
% ======================================================================
\documentclass[german]{latex4ei/latex4ei_sheet}
\usepackage[european]{circuitikz}
\usepackage{tabularx}
\usepackage{amsmath}
\usepackage{xcolor}
\usepackage{multirow}
\usetikzlibrary{arrows, calc, intersections}
\usepackage{calc}

% Für code
\definecolor{COMMENTGREEN}{HTML}{228B22}
\definecolor{MATLABBACKGROUND}{HTML}{FCFCDC}
\lstset{ %
language=Matlab,						% choose the language of the code
%basicstyle=\ttfamily,					% the size of the fonts that are used for the code
%emphstyle=\color{yellow}\ttfalily,
%keywordstyle=\color{blue}\ttfamily,
%stringstyle=\color{magenta}\ttfamily,
%commentstyle=\color{COMMENTGREEN}\ttfamily,
xleftmargin=10.3pt,						% distance to margin left
xrightmargin=-3pt,						% distance to margin right
%linewidth=\widthof{\begin{sectionbox}}}		% this would be better instead of left and right margin
%aboveskip=0.6\baselineskip,
%belowskip=0\baselineskip,
numbers=left,                   		% where to put the line-numbers
%framexleftmargin=1.5em,					% distance to xleftmargin
%numberstyle=\ttfamily\footnotesize,		% the size of the fonts that are used for the line-numbers
%stepnumber=1,							% the step between two line-numbers. If it is 1 each line will be numbered
%numbersep=5pt,							% how far the line-numbers are from the code
%backgroundcolor=\color{MATLABBACKGROUND},	% choose the background color. You must add \usepackage{color}
%showspaces=false,						% show spaces adding particular underscores
%showstringspaces=false,					% underline spaces within strings
%showtabs=false,							% show tabs within strings adding particular underscores
%frame=single,							% adds a frame around the code (Box) (Für top und bottom rule set option to "lines")
%rulecolor=\color{gray},					% color of framebox rule
%tabsize=4,								% sets default tabsize to 2 spaces
%captionpos=b,							% sets the caption-position to bottom
breaklines=true,						% sets automatic line breaking
breakatwhitespace=false,				% sets if automatic breaks should only happen at whitespace
escapeinside={\%*}{*)}					% if you want to add a comment within your code
}

% tabularx definition
\newcolumntype{C}{>{\centering\arraybackslash}X}
\newcolumntype{L}{@{\extracolsep\fill}X}

% SI-Zahlen mit Komma als Dezimaltrenner
\sisetup{locale=DE}

% SI-Einheiten
\DeclareSIUnit\voltampere{VA}
\DeclareSIUnit\var{Var}
\DeclareSIUnit\newtonmeter{Nm}
\DeclareSIUnit\voltsecond{Vs}
\DeclareSIUnit\amperesecond{As}

\DeclareMathOperator*{\argmin}{arg\,min}
\DeclareMathOperator{\cond}{cond}
\DeclareMathOperator{\rang}{rang}

\DeclareMathOperator{\Bild}{Bild}
\DeclareMathOperator{\defect}{def}

% Title
\title{T3p: Elektrodynamik}
\author{Marcus Müller}
\myemail{marcus.mueller@physik.uni-muenchen.de}
\mywebsite{}

% Dokumentbeginn
% ======================================================================
\begin{document}

\IfFileExists{git.id}{\input{git.id}}{}
\ifdefined\GitRevision\mydate{\GitNiceDate\ (git \GitRevision)}\fi

\maketitle
\section{Mathematische Grundlagen}


	\begin{itemize}
		\item \textbf{Wie lauten Skalar- und Vektorprodukt in Indexnotation?} 
		\begin{itemize}
			\item Skalarprodukt: $\vec{x}\cdot\vec{y}=x^iy_i$ 
			\item Vektorprodukt/Kreuzprodukt: $\vec{x}\times\vec{y}=\varepsilon_{ijk}a_ib_j\vec{e}_k$
		\end{itemize}
		\item \textbf{Was ist das Ergebnis von $\partial_ir_j$?}
		\begin{equation}
			\partial_ir_j=\left(
			\begin{matrix}
  				\partial_1r_1 & ... &\partial_1r_m \\
 				... &... & ...\\
 				\partial_nr_1 &... & \partial_nr_m
			\end{matrix}\right)
		\end{equation}
		\item \textbf{Was ergibt die Kontraktion $\varepsilon_{ijk}\varepsilon_{klm}$ und wie können Sie sich diese merken?}
		\begin{equation}
			\varepsilon_{ijk}\varepsilon_{klm}=\delta_{jl}\delta_{km}-\delta_{km}\delta_{kl}
		\end{equation}
		
		\item \textbf{Wie definieren Sie die Menge, die eine Kugel K um einen\\
		 beliebigen Punkt beschreibt? Wie definieren sie deren Rand $\partial K$? Wie lauten die zugehörigen Parametrisierungen}\\
		$K: (x−x_M)^2+(y−y_M)^2+(z−z_M)^2=r^2$
		\colorbox{red}{noch bearbeiten}
		\item \textbf{Wie lauten die Mengen und die Parametrisierungen der letzten Frage für einen Zylinder Z um eine beliebige zu $e_z$ parallele Achse? Wie lauten sie für eine zur xy-Ebene parallele Kreisscheibe S mit beliebigem Zentrum?}\\
		\colorbox{red}{noch bearbeiten}
		\item \textbf{Was ist die Bedeutung der Divergenz und der Rotation eines Vektorfeldes?}
			\begin{itemize}
				\item Divergenz: $\nabla\vec{r}$ $\Rightarrow$ Stärke des Auseinanderstrebens
				\item Rotation: $\nabla\times\vec{r}$ $\Rightarrow$ ordnet einem Vektorfeld ein neues Vektorfeld zu
			\end{itemize}
		\item \textbf{Wie konstruiert man Gradient, Divergenz, Rotation und Laplace-Operator in
krummlinigen Koordinaten?}\\
		\\\colorbox{red}{noch bearbeiten}
		\item \textbf{Wie berechnet man ein skalares bzw. vektorielles Kurvenintegral?}
		\begin{itemize}
			\item skalares Kurvenintegral: $\int_\gamma fds:=\int_a^bf(\gamma(t))\cdot|\dot{\gamma}(t)|dt$
			\item vektorielles Kurvenintegral: $\int_\gamma \vec{f}d\vec{s}:=\int_a^b\vec{f}(\vec{\gamma}}(t))\cdot|\dot{\vec{\gamma}}(t)|dt$
		\end{itemize}
		\item \textbf{Wie berechnet man ein skalares bzw. vektorielles Flächenintegral in kartesischen, Zylinder- oder Kugelkoordinaten?}
		\\\colorbox{red}{noch bearbeiten}
		\item \textbf{Wie lauten der Satz von Gauß und der Satz von Stokes?}
			\begin{itemize}
				\item Satz von Gauß: $\int_\nu d^3x(\nabla\cdot\vec{A})=\int_{\partial\nu}d\vec{f}\cdot\vec{A}$
				\item Satz von Stokes: $\int_{\partial F}d\vec{f}\cdot(\nabla\times\vec{A})=\int_{\partial F}d\vec{r}\cdot\vec{A}$
			\end{itemize}
		\item \textbf{Was sind die Eigenschaften der Delta-Distribution?}
			\begin{enumerate}
				\item $\delta(x-c)=
				\begin{cases}
					0 & x\neq c\\
					\infty & x=c
				\end{cases}$	
				\item $\int_{-\infty}^\infty\delta(x-c)dx=1$
			\end{enumerate}
		\item \textbf{Wie steht der Laplace-Operator mit der Delta-Distribution in Verbindung?}
		\\\colorbox{red}{noch bearbeiten}
		\item \textbf{Was besagt das Helmholtz Theorem?}\\
			Es ist Möglich, ein (fast) beliebiges Vektorfeld als Superposition eines rotationsfreien (wirbelfreien) Feldes und eines divergenzfreien (quellenfreien) Feldes darzustellen
\end{itemize}	
\section{Maxwell-Gleichungen}
\begin{itemize}
	\item \textbf{Wie lauten die Maxwell-Gleichungen im cgs-System?}\\\colorbox{red}{noch bearbeiten}
	\item \textbf{Wie leitet man die Maxwell-Gleichungen in integraler Form her?}\\\colorbox{red}{noch bearbeiten}
	\item \textbf{Was ändert sich bei den Gleichungen in Materie? Was ist die Bedeutung der neu eingeführten Größen?}\\\colorbox{red}{noch bearbeiten}
	\item \textbf{Was ist der Unterschied zwischen freier und gebundener Ladungsdichte? Wie berechnet man letztere?}\\\colorbox{red}{noch bearbeiten}
	\item \textbf{Wie leitet man die Kontinuitätsgleichung her und was beschreibt sie?}\\\colorbox{red}{noch bearbeiten}
	\item \textbf{Wie formuliert man Ladungs- und Stromdichte für ein gegebenes System?}\\\colorbox{red}{noch bearbeiten}
	\item \textbf{Was beschreibt der Poynting-Vektor?}\\\colorbox{red}{noch bearbeiten}
	\item \textbf{Wie lautet die Impulserhaltung im elektromagnetischen Feld? Was ist die Bedeutung der einzelnen Terme?}\\\colorbox{red}{noch bearbeiten}
	\item \textbf{Was besagt der Satz von Poynting? Wie lautet dessen integrale Form?}\\\colorbox{red}{noch bearbeiten}
	\item \textbf{Wie berechnet man die Feldenergie des elektromagnetischen Feldes?}\\\colorbox{red}{noch bearbeiten}
	\item \textbf{Wie kann man das elektrische und magnetische Feld mit Hilfe von Potentialen ausdrücken?}\\\colorbox{red}{noch bearbeiten}
	\item \textbf{Was ist eine Eichtransformation und wodurch kommt sie zustande?}\\\colorbox{red}{noch bearbeiten}
	\item \textbf{Wie transformieren sich elektrische und magnetische Felder unter Zeit- bzw. Raumspiegelung?}\\\colorbox{red}{noch bearbeiten}
	\item \textbf{Wie kann man die physikalischen Größen der Elektrodynamik zwischen cgs- und SI-System in Verbindung setzen? Was beschreibt die Einheit 1esu?}
\end{itemize}
\section{Elektrostatik}
\section{Magnetostatik}
\section{Spezielle Relativitätstheorie}
\section{Elektromagnetische Wellen}
\tiny{based on template from www.latex4ei.de}

\end{document}
